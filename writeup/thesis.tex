\documentclass[final]{ukthesis}
%you must include these 2 packages.
\usepackage[pdfauthor={Jack Bandy},
            pdftitle={The Title},
            pdfsubject={The Subject},
            pdfkeywords={Some Keywords},
            pdfproducer={Latex with hyperref},
            pdfcreator={latex->dvips->ps2pdf},
            pdfpagemode=UseOutlines,
            bookmarksopen=true,
            letterpaper,
            bookmarksnumbered=true]{hyperref}
\usepackage{memhfixc}
%%%%%%%%%%%%%%%%%%%%%%%%%%%%%%%%%%%%%%%%%%%%%%%%
\begin{document}
%author data
\author{Jack Bandy}
\title{INTERACTIVE MACHINE LEARNING FOR WORD RECOGNITION ON DAMAGED HANDWRITTEN DOCUMENTS}
\abstract{an abstract}
\advisor{your advisor}
\keywords{keywords go here}
\dgs{DGS name here}
%the title pages
\frontmatter
\maketitle
\begin{acknowledgments}
Acknowledge people/things here
\end{acknowledgments}
\begin{dedication}
Dedicated to things (optional)
\end{dedication}
\tableofcontents\clearpage
\listoffigures\clearpage
\listoftables\clearpage
%----------------------------------------------
\mainmatter
\chapter{Background}
\section{Project Components}
There are two main components of the project. The first is a semi-supervised machine learning approach to document transcription, and the second is a word tracing tool for textual scholarship.

\subsection{An Interactive Approach to Automated Transcription}
Automated transcription is ideal and sometimes necessary for larger datasets of handwritten documents. Automated transcription for printed documents as well as handwritten documents is now on par with human performance. However, damaged historical documents present unique challenges.

I develop the following approach: given a small set of labeled samples, train a neural network in a semi-supervised manner using both labeled and non-labeled data. Once the initial model is trained, use it to create a transcription of the full document. During the transcription process, the model keeps track of difficult word images, prioritizing them for manual labeling afterwards.

\subsection{Word Tracing}
Once the transcription of a document is generated, many scholars wish to trace the outputted text back to the original manuscript image. Building on state-of-the-art word spotting techniques, I implement a tool that traces transcript text back to the original input image so that scholars can easily navigate and visualize transcriptions.

\section{Literature Review}
\subsection{2009}
\begin{itemize}
\item Finding words in alphabet soup: Inference on freeform character recognition for historical scripts \cite{howe2009finding}.
\end{itemize}

\subsection{2012}
\begin{itemize}
\item A novel word spotting method based on recurrent neural networks \cite{frinken2012novel}.
\item End-to-end text recognition with convolutional neural networks \cite{wang2012end}.
\end{itemize}

\subsection{2013}
\begin{itemize}
\item Handwritten word recognition using mlp based classifier: A holistic approach \cite{acharyya2013handwritten}.
\item Feature extraction with convolutional neural networks for handwritten word recognition \cite{bluche2013feature}.
\end{itemize}

\subsection{2014}
\begin{itemize}
\item A combined system for text line extraction and handwriting recognition in historical documents \cite{fischer2014combined}
\end{itemize}

\subsection{2015}
\begin{itemize}
\item Efficient segmentation-free keyword spotting in historical document collections \cite{rusinol2015efficient}.
\item Adapting off-the-shelf cnns for word spotting \& recognition \cite{sharma2015adapting}.
\item Segmentation-free handwritten Chinese text recognition with LSTM-RNN \cite{messina2015segmentation}.
\end{itemize}

\subsection{2016}
\begin{itemize}
\item On the Benefits of Convolutional Neural Network Combinations in Offline Handwriting Recognition \cite{suryani2016benefits}.
\item Reading text in the wild with convolutional neural networks \cite{jaderberg2016reading}.
\item PHOCNet: A deep convolutional neural network for word spotting in handwritten documents \cite{sudholt2016phocnet}.
\item SpottingNet: Learning the Similarity of Word Images with Convolutional Neural Network for Word Spotting in Handwritten Historical Documents \cite{zhong2016spottingnet}.
\end{itemize}

\subsection{Surveys}
\begin{itemize}
\item A survey of document image word spotting techniques \cite{giotis2017survey}.

\item A survey on handwritten documents word spotting \cite{ahmed2017survey}.
\end{itemize}
\copyrightnotice

\chapter{The First Chapter}
\section{The First Section}
Math goes here.
\begin{figure}[h]
\centering
Here's a figure
\caption{A Simple Figure}
\end{figure}
\begin{table}[h]
\centering
\begin{tabular}{c|c}
Here & is \\
\hline
a & table
\end{tabular}
\caption{A Simple Table}
\end{table}
\copyrightnotice
%-----------------------------------------------
\backmatter
\bibliographystyle{unsrt}   % this means that the order of references
			    % is dtermined by the order in which the
			    % \cite and \nocite commands appear
\bibliography{mybib}  % list here all the bibliographies that
			     % you need. 

\chapter{Vita}
A brief vita goes here.
\end{document}